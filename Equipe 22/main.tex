\documentclass[12pt,a4paper,onecolumn,titlepage]{article}

\usepackage[brazil]{babel}
\usepackage[utf8]{inputenc}
\usepackage{graphicx}


\begin{document} %Inicio do documento

\begin{titlepage} %Capa
	
	\vfill
	\begin{center}
	
		{\large \textbf{Faculdade de Ciências e Tecnologia\\Universidade Estadual Paulista\\``Júlio de Mesquita Filho''}} \\[3cm]
		{\large \textbf{Darlan Murilo Nakamura de Araújo\\Pietro Barcarollo Schiavinato}}\\[4cm]
		{\Large Documento de Requisitos}\\
		{\Large Insira o nome aqui}\\[4cm]


	\vfill
	\vspace{1.5cm}
	
	\large \textbf{Presidente Prudente\\}
	\large \textbf{Maio - 2017}
	
	\end{center}
	
\end{titlepage}
%Fim da capa
\newpage

\renewcommand{\contentsname}{Índice}
\tableofcontents
\newpage
%Conteúdo do documento
\section{Introdução}
\label{sect:intro}

\subsection{Propósito do documento de requisitos}

O documento tem como objetivo descrever as perspectivas do produto e suas
respectivas funcionalidades, bem como todos os recursos de hardware e software necessários para a elaboração deste, contendo todas as especificações de
requisitos necessários. Esta ferramenta visa prover a emissão de boletos e a administração dos boletos gerados.

\subsection{Escopo do produto}

Esta ferramenta busca oferecer um ambiente de forma a fornecer o serviço de geração de boleto e de administração dos boletos gerados, de forma que o administrador possa ter o controle dos boletos gerados.

\subsection{Definições, siglas e abreviaturas}

Segue abaixo, a lista de siglas e abreviaturas utilizadas neste documento:
\begin{itemize}
\item RF: Requisito Funcional
\item RNF: Requisito não Funcional
\end{itemize}

\subsection{Visão geral do documento}

Neste documento serão abordados alguns assuntos como divididos a seguir:
na Seção 2 é apresentado a descrição geral do produto, assim como suas
funcionalidades e respectivas restrições bem como o ambiente de execução.
A seguir na Seção 3 é descrito o conjunto de todos os requisitos do sistema
e do projeto.

\section{Descrição geral}
\label{sect:descricao}

\subsection{Perspectiva do Produto}

O sistema utiliza um servidor que trabalha diretamente com um banco de dados relacional, o qual armazena os usuários e os dados referentes ao boleto.


\subsection{Funções do Produto}
O sistema deve permitir que os usuários possam gerar boletos para o banco de sua escolha. Ao usuário administrador deve-se permitir gerir os boletos gerados, dar baixa em boletos pagos, e reemitir boletos vencidos.

\subsection{Características do Usuário}

O usuário comum tem como característica o interesse em gerar boletos para o banco de sua escolhe. O usuário do tipo Administrador tem como característica o interesse em gerir os boletos gerados. 


\subsection{Restrições}

O sistema será desenvolvido para plataforma web, desta forma sendo acessível tanto por desktop ou por dispositivos móveis, sendo que este último tem maiores restrições de hardware, porém a execução da plataforma deve ocorrer
normalmente em ambas as plataformas. Será desenvolvido em HTML 5, CSS 3 e JavaScript, logo quaisquer navegadores que não fornecerem suporte completo a estas tecnologias não serão suportados pela ferramenta.\\

\subsection{Suposições e Dependências}

Este software não possui nenhuma limitação adicional ao que citado na sessão 2.4 quanto ao sistema operativo ou hardware .

\section{Requisitos Específicos}
\label{sect:requisitos}

\subsection{Requisitos Funcionais}

\begin{description} 


\item	RF 1.1 - O sistema deve permitir ao usuário gerar boletos para o banco de sua escolha (E)
\item	RF 1.2 - O sistema deve permitir ao usuário Administrador dar baixa em boletos (E)
\item	RF 1.3 - O sistema deve permitir ao usuário Administrador visualizar os usuários que emitiram o boleto (E)
\item	RF 1.4 - O sistema deve permitir ao usuário Administrador reemitir boletos vencidos (E)

\end{description}

\subsection{Requisitos Não Funcionais}


\begin{description}
\item RNF 1.1 - O sistema será desenvolvido para plataforma web utilizando as  linguagens HTML, CSS e Javascript, utilizando requisições AJAX. Para
		a parte de servidor será utilizada a linguagem PHP e para a persistência
		será utilizado o banco de dados relacional MySQL.


\end{description}

%\subsection{Requisitos Desempenho}
%\subsection{Requisitos Lógicos de Banco de Dados}
%\subsection{Restrições de Projeto}
%\subsection{Atributo do Sistema de Software}



%\subsection{Organização}
%\section{Informações de Apoio}
%\label{apoio}
  
  
%\subsection{Índice}
%\label{indice}

\newpage

\renewcommand{\contentsname}{Índice}


%\section{Apêndices}
%\label{sect:apendices}

%Fim do conteúdo do documento
%referencias - estilos: http://www.cs.stir.ac.uk/~kjt/software/latex/showbst.html
%\bibliographystyle{acm}
%\bibliography{referencias}

\end{document} %Fim do documento
